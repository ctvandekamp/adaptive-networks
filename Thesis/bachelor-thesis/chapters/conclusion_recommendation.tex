\chapter{Conclusions and recommendations}

In this thesis, we considered adaptive network models with discrete and continuous state sets. The 2-state adaptive network models were analysed analytically and have two stationary points: an ordered and a disordered solution. The ordered solution, in which all states are equally distributed over all nodes, is stable for a relatively high dimensionless noise ratio compared to the dimensionless link creation ratio. The disordered solution, where one of the two states has a higher density than the other, is stable for relatively low dimensionless noise ratios. The transition from the ordered to the disordered state occurs through a supercritical pitchfork bifurcation. Numerical solutions are in line with these analytic results. Hence, irrespective of what biological, physical or social system the adaptive network is applied to, we know the outcome once the noise and link creation ratios are known. 

Furthermore, an adaptive network model was derived that works on a periodic or non-periodic continuous state set, obeying comparable dynamical rules as in the considered discrete state models. This model consists of a non-closed system of two coupled partial integro-differential equations, which can be analysed either in the mean field approximation or in the moment closure approximation. 

Both approximations were applied to a swarming system, consisting of two-dimensional motion of self-propelled particles with constant speed. Numerical solutions of the mean field model show that the disordered solution forms a stationary distribution if the noise rate is relatively high compared to the three-body interaction rate. For lower noise rates, the system ends up in an ordered solution, of which the variance is related to the system parameters by a square root function. A Cauchy distribution fits this ordered stationary solution very well, although it does not form an analytic stationary solution to the PDEs. Future research could focus on finding an analytic stationary ordered solution using more sophisticated PDE techniques.  

The moment closure approximation model was also numerically solved. Again, for relatively high noise rates compared to the three-body interaction rate the system ends up in a disordered solution. If the noise rate is relatively low, the system ends up in an ordered distribution, of which the variance does not spread as much as in the mean field approximation models and is probably not given by a square root function of the system parameters. There exists a bistable region in which the final distribution depends on the initial conditions of the system. This a sign of a subcritical pitchfork bifurcation in combination with a saddle-node bifurcation. Further research is needed to find the exact details of these bifurcations. Moreover, the current script solving the system numerically could be optimised, such that higher resolution solutions can be found for more different system parameters. These solutions could help in determining the exact form of the stationary solutions. Further research could also focus more on analysis the link distribution function and exploring the effect of different link creation and deletion rates and more sophisticated initial conditions. 

The comparison between continuous state adaptive networks, agent-based models and real-life swarming systems is still to be made. It will be very interesting to see whether the derived model can be an alternative to already existing models and what advantages and disadvantages there are. Moreover, there might be applications other than swarming systems that can be modelled very accurately as a continuous state adaptive network.

In the future, multiple possible extensions to the derived adaptive network model can be made. For example, one could do research into the effects of changing the interaction (rates) depending on the state a certain node occupies. These models may be a better representation of certain real-life situations. Another possibility is extending the concepts of adaptive networks with a stochastic component. In that case one would not be able to determine the complete time evolution beforehand.



%\textbf{Recommendations}

%\todo{arbitrary order, should be typed out}
%\begin{enumerate}
%	\item One could try finding analytic (stationary) solutions to the continuous model using more sophisticated PDE techniques.
%	\item The accuracy of the numerical integration scheme could be improved by implementing a more accurate discrete integration method, such as the trapezoidal rule. 
%	\item Significant improvements can be made in making the code more efficient. Indicate why this is necessary!
%	\item One could extend the concepts of adaptive networks with a stochastic component. With the described models, the complete time evolution is exactly known beforehand. This way, states which are initialised as a minority may end up in a majority.
%	\item One could do research into the effects of changing the interaction rates depending on the state a certain node occupies. That is, for example, nodes with a state in subset $S \subset \Omega$ have different rates (or obey completely different interactions) than nodes with a state in the complement $\Omega \setminus S$
%	\item One could do more research into the derived continuous state adaptive network model, for example how do more sophisticated initial distributions evolve. This might be done by deriving and analysing the change in the mean of the distribution
%	\begin{equation}
%	\frac{d}{dt} \langle x \rangle = \frac{\partial}{\partial t} \int_{-\pi}^\pi dx\ x\ f(x;t)
%	\end{equation}
%	and by deriving and analysing the change of the second moment
%	\begin{equation}
%	\frac{d}{dt} \langle x^2 \rangle = \frac{\partial}{\partial t} \int_{-\pi}^\pi dx\ x^2\ f(x;t),	
%	\end{equation}
%	which is a measure for the order in the system and the width of the distribution for various initial conditions.
%	\item One could compare the results to real-life swarming systems and agent-based models to check whether the continuous state adaptive network models might be better models than the current discrete state adaptive network models.
%	\item Applying continuous state adaptive networks to other systems than swarming systems.
%\end{enumerate}
%
		 
%\todo{numerieke probleem in het script zou zomaar eens niet oplosbaar kunnen zijn. De resultaten zijn opzich al goed nu. Het script preciezer maken kan zorgen dat de instabiliteit langer uitgesteld wordt. Dit kunnen bv dingen zijn zoals de trapezoidal rule gebruiken voor integratie en switchen naar float128, meer grid points, optimale thresholdwaarde vinden voor afkappen: zo klein mogelijk maar niet te klein dat we problemen krijgen etc. Of misschien overstappen naar een compleet ander numeriek schema. Er zou bijvoorbeeld gekeken kunnen worden naar of Newton-Raphson een oplossing biedt om de steady state oplossingen te vinden.}

