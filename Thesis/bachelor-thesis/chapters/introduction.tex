\chapter{Introduction}

A flock of birds is a fascinating phenomenon. Thousands of birds that fly in certain patterns, collectively, and yet there is no real leader. No bird tells the other birds where to fly. The question that naturally arises is: how do birds coordinate their movement and how do they know at what speed they should fly in which direction? At the beginning of the previous century, researchers studying this phenomenon proposed the concept of a `group mind' in which the nervous systems of individual birds are connected \cite{Selous1931}. Some decades ago, many researchers found the idea that there must be some kind of a leader more plausible \cite{Heppner1990}, but no such leader has been found in detailed studies \cite{Pomeroy1983}. 

Many more of these processes of so-called collective motion take place in nature, such as schools of fish, swarms of insects and herds of certain mammals. Recent studies into these kinds of swarming systems  demonstrated the existence of a set of universal organising principles that all swarming systems have in common \cite{Huepe2011}. However, the aforementioned question remained unanswered. Nowadays, to find the answer, most theoretical studies of collective motion represent a swarm either as a continuous medium \cite{Toner1998} or focus on (sophisticated) agent-based models of a system of self-propelled particles \cite{Huepe2011, VanDrongelen2015, Vicsek1995, Vicsek2012}. The latter obeying certain dynamical rules that facilitate self-organisation. 

Opinion formation processes in human populations form another class of systems in which the collective decision making is studied. Many similarities between these voter models and swarming systems can be identified \cite{Huepe2011}, but most of the time these processes are modelled differently. In the voter models, the human population is represented as a network. The nodes of the network correspond to the people in the population and nodes corresponding to individuals that speak on a regular basis to each other are connected by a link \cite{Sood2005}.

In 2011, the idea of using the networks of the voter model to model the real-life swarming system of one-dimensional movement of locusts \cite{Buhl2006} was proposed by Huepe \textit{et al.} \cite{Huepe2011}. The network nodes, representing the locusts, have a binary internal state, which corresponds to the direction of movement of the locust. Nodes corresponding to individuals which are aware of each other's direction of movement are connected by a link. There is no variable that keeps track of the position of the locust in physical space. Within this configuration, certain dynamical rules are imposed that give individuals the possibility of changing their internal state randomly or of aligning themselves with neighbouring nodes. The model is called an \textit{adaptive} network model since the dynamics allow for the creation and deletion of links, which causes a changing network topology over time. This system turned out to reproduce multiple characteristic features of swarms, amongst others the existence of an ordered state, corresponding to collective motion, and a disordered state, in which all locusts move randomly. Furthermore, the results suggested that keeping track of the spatial position of each individual explicitly could be of minor importance in obtaining self-organisation in such systems.

A couple of years later Chen \textit{et al.} \cite{Chen2016} derived and analysed a class of adaptive network models in which the internal state of a node was not binary, but chosen from a state set containing a finite number of discrete states. Such a state set is convenient since swarms moving in either two- or three-dimensional space can be modelled with these networks. 

This text will focus on introducing the reader to this swarming systems class of adaptive network models. Moreover, we derive an adaptive network model that works on a continuous state set. For these models, the set of directions an individual can choose from does not have to be discretised. Hence this may lead to more accurate models.

In chapter two, we will familiarise the reader with discrete state adaptive network models and the dynamical rules that allow for self-organisation in the system. In the third chapter, the 2-state (binary) adaptive network model is analysed analytically. Moreover, we will perform a bifurcation analysis. Subsequently, we formally derive the continuous state adaptive network models in chapter four. The application of the derived model to swarming systems, including discussion of the results, is contained in chapter five. Finally, chapter six presents the conclusions in combination with recommendations for further research. 

This work is part of the bachelor programmes Applied Mathematics and Applied Physics, provided by the faculties Electrical Engineering, Mathematics and Computer Science and Applied Sciences at the Delft University of Technology. It constitutes a bachelor thesis in both study programmes simultaneously. 