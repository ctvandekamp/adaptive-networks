\chapter{Transformation to a continuous state set}
So far we have considered adaptive network models with a discrete state set $\Omega$, governed by a set of four dynamical rules. Recently there have been quite some publications on comparable discrete state (adaptive) network models, ranging from very fundamental mathematical backgrounds to applications of discrete state adaptive networks, for example \cite{Bauch2002,Chen2016,Demirel2014,Gross2006,Huepe2011,Keeling1999,Kimura2008,Newman2003,Sayama2013,Zschaler2012}. 
These models have however some drawbacks. For instance, if the network is used a model for self-organisation in a two-dimensional swarming system consisting of self-propelled particles, one could use $\Omega=\{\text{up}, \text{down}, \text{left}, \text{right}\}$. This means that diagonal motion would not be a separate state. A global diagonal movement would certainly be possible, but looking at a small time scale this movement would consist of small steps in two directions that are contained in $\Omega$. The problem with this is that the dynamics happen on the same small time scale as this flipping in states, such that a particle with a global diagonal movement to the north-west would interact half of the time similar with a particle moving north as a particle moving west would.

In this chapter, we make a transformation of discrete state networks to work on a continuous state set. The goal of making this transformation is to see if these networks might be a better model of reality in cases such as the aforementioned example. So far, limited research has been done into these type of networks. We derive the system of equations describing this class of models in a general form. 

This will be the most technical chapter of this thesis. The reader can skip ahead to \cref{chapter:swarming_systems} for the results. There, we will apply the derived model to self-organisation in two-dimensional swarming systems, where we have $\Omega = (-\pi,\pi]$ representing directions of 2D movement of self-propelled particles.

 
\section{State and link dynamics}
The continuous state adaptive network model will be derived in a general form. However to stay specific, some pre-determined properties will be applied to the model. The discrete state model is used as a starting point and from here we will make adaptations to make the model work on a continuous state set. We will write lower case letters for states contained in a continuous state set instead of upper case letters that are used for discrete states.  Starting with the types of dynamics, the dynamical rules used in the discrete state systems are not suitable anymore, hence they should be altered.  Especially the second type of state dynamics, where a triplet of nodes $y-x-y$ switches to state $y-y-y$ with rate $\sigma_d$ is not properly defined if we switch to continuous states; the probability for a node having two neighbours in the exact same state will be zero. Hence, the four types of interactions will be redefined for the continuous model:

\begin{description}
    \item [Type 1] Nodes change to another uniformly chosen state with rate $\eta$.
    \item [Type 2] Nodes adopt to the average state of two neighbours with rate $\sigma_c$. 
    \item [Type 3] Links are created between arbitrary not-linked nodes with rate $\alpha$.
    \item [Type 4] Links are removed between two arbitrary linked nodes with rate $\beta$.
\end{description}


\begin{figure}[t]
	\centering
	\begin{tikzpicture}[rotate=0]
	
	\draw[] (-5,0) node[circle, minimum size =0.5cm, draw] (state11) {};
	\draw[] (-3,0) node[circle, fill=black, minimum size =0.5cm, draw] (state12) {};
	
	\draw[->, shorten <=3pt, shorten >=3pt] (state11)--(state12) node[above, xshift=-1cm] {$\eta$};
	\draw[very thick] (state11)--(-5.4,0.8);
	\draw[very thick] (state11)--(-5.4,-0.8);
	\draw[very thick] (state12)--(-3.4,0.8);
	\draw[very thick] (state12)--(-3.4,-0.8);
	
	%%%%%%%%%%%%%%%%
	
	
	\draw[] (0,0) node[circle, pattern=soft crosshatch, minimum size =0.5cm, draw] (state21) {};
	\draw[] (-0.6,-1) node[circle, minimum size =0.5cm, draw] () {};
	\draw[] (1.4,-1) node[circle,minimum size =0.5cm, draw] () {};
	\draw[] (2,0) node[circle, fill=gray, minimum size =0.5cm, draw] (state22) {};
	\draw[] (-0.6,1) node[circle, fill=black, minimum size =0.5cm, draw] () {};
	\draw[] (1.4,1) node[circle, fill=black,minimum size =0.5cm, draw] () {};
	
	\draw[->, shorten <=3pt, shorten >=3pt] (state21)--(state22) node[above, xshift=-1cm] {$\sigma_c$};
	\draw[very thick] (state21)--(-0.45,-0.8);
	\draw[very thick] (state21)--(-0.6,1);
	\draw[very thick] (state22)--(1.55,-0.8);
	\draw[very thick] (state22)--(1.4,1);
	
	%%%%%%%%%%%%%%%%%%
	
	\draw[] (7,0.583) node[circle, fill=black, minimum size =0.5cm, draw] (link11) {};
	\draw[] (5,0.583) node[circle, fill=black, minimum size =0.5cm, draw] (link12) {};
	\draw[] (7,-0.583) node[circle, minimum size =0.5cm, draw] (link13) {};
	\draw[] (5,-0.583) node[circle, minimum size =0.5cm, draw] (link14) {};
	
	\draw[very thick] (link11)--(link13);
	
	\draw[->, shorten <=3pt, shorten >=3pt, transform canvas={yshift=-0.683cm}] (link11)--(link12) node[below, xshift=1cm] {$\beta$};
	\draw[->, shorten <=3pt, shorten >=3pt, transform canvas={yshift=-0.483cm}] (link12)--(link11) node[above, xshift=-1cm] {$\alpha$};
	
	\end{tikzpicture}
	\caption{Illustration of the model, four types of dynamics are applied to the continuous state adaptive network models.  The internal state of each node (circle) is represented by its colour. The grey colour corresponds to the average of the states black and white. The chequered pattern indicates a random state in the state set $\Omega$. These dynamics take place irrespective of any additional links that may be present, but are not drawn.}
	\label{fig:dynamics_continuous}
\end{figure}
The dynamics take place irrespective of any additional links that may be connected to a node. Furthermore, we cannot use the same definition of densities for nodes and small subgraphs as before. These discrete network moments would not be properly defined if a system of continuous states is considered. Therefore state and link density functions will be introduced. These functions describe the distribution of the possible states in the network (comparable to probability density functions in probability theory). In order to introduce these formally, let us first consider the cumulative distribution functions $F(x;t)$ and $L_n(x_1,x_2,...,x_n;t)$ and assume that the state set $\Omega$ is a single non-empty, bounded real interval. That is $\Omega = [a,b]$, or $\Omega = (a,b)$, where $a,b \in \mathbb{R}$ and the endpoints are either included in or excluded from the interval. 
\begin{definition}
	The cumulative distribution function (CDF) $F(x;t)$ denotes the density of nodes that have a state in $[a, x] \subseteq \Omega$ at time $t$. 
\end{definition}
\begin{definition}
	Let $n\in\mathbb{N}\setminus\{1\}$. For $n$-body subgraphs (e.g. links, triplets etcetera) the $(n-1)$'th order moment cumulative distribution function $L_n(x_1,x_2,...,x_n;t)$ is the density of motifs in which the first node occupies a state in $[a, x_1]$, which is connected to a second node occupying a state in $[a, x_2]$, etcetera, at time $t$.
\end{definition}

CDFs can be defined analogously for open intervals. The node density, but also the higher order densities are all normalised against the total number of nodes $N$. The density of nodes in a certain state is a zeroth-order moment, the density of e.g. $x-y$-links a first-order moment and the density of subgraphs consisting of $n$ nodes an $(n-1)$'th-order moment. Subsequently, the state and link density distribution functions can be defined as follows.  
\begin{definition}
	The state distribution function $f(x;t) : \Omega \times \mathbb{R}_{\ge0} \to  \mathbb{R}_{\ge0}$ is the unique function that satisfies $F(x;t) = \int\limits_a^x d\bar{x}\ f(\bar{x};t)$.
\end{definition}
\begin{definition}
	The $(n-1)$'th order moment distribution function $l_n(x_1,x_2,...,x_n;t): \Omega^n \times \mathbb{R}_{\ge0} \to  \mathbb{R}_{\ge0}$ is the unique function that satisfies $L_n(x_1,x_2,...,x_n;t) = \int\limits_a^{x_{1}} d\bar{x_1} \int\limits_a^{x_{2}} d\bar{x_2}\ ... \int\limits_a^{x_n} d\bar{x_n}\  l(\bar{x_1}, \bar{x_2},...,\bar{x_n};t)$.
\end{definition}
In these definitions, $\bar{x}$ and $\bar{x_i}, i \in \{1,2,...,n\}$, are used as integration variables. Note that the link distribution functions obey a certain symmetry. By definition $l_2(x,y;t) = l_2(y,x;t)$, which makes sense since if an $x$-node is connected to a $y$-node, one can describe the links both as an $x-y$- or as an $y-x$-link. Moreover, we have $l_3(x,y,z;t) = l_3(z,y,x;t)$, both representing the density of $x-y-z$-triplets.  Furthermore, we remark that the distribution functions do not have to be continuous in state space. However, in case a distribution function is continuous in state space, we have $f(x;t) = \frac{\partial}{\partial x} F(x;t)$ and $l_n(x_1,x_2,...,x_n;t) = \frac{\partial}{\partial x_1} \frac{\partial}{\partial x_2} ... \frac{\partial}{\partial x_n} L(x_1, x_2,...,x_n)$. Intuitively $f(x;t)dx$ is the density of nodes occupying a state within the interval $[x,x+dx]$, while $ l_n(x_1, x_2,...,x_n;t)dx_1dx_2...dx_n$ is the density of $n$ body subgraphs with the first node in state $[x_1,x_1+dx_1]$ is connected to a second node in state $[x_2,x_2+dx_2]$, etcetera. 

Besides, it will be useful to derive a density distribution function for subgraphs in configuration $^wx^y_z$, with the middle node in state $x$ connected to nodes in states $w,y$ and $z$. 
\begin{definition}
	The third order moment distribution $l(^wx^y_z;t): \Omega^4 \times \mathbb{R}_{\ge0}  \to \mathbb{R}_{\ge0}$ describes the density of subgraphs in configuration $^wx^y_z$ at time $t$.
\end{definition}
We will derive the equations describing the effect of the four types of dynamics on the state and link distributions in the most general form possible, such that they can be applied to model a great variety of phenomena. To keep things clear the equations will be introduced for each type of dynamics separately. The effect of the dynamics of type 1 on the state density function $f(x;t)$ are captured by the following partial differential equation (PDE). The superscript $(1)$ indicates that only the dynamics of the first type are captured by this equation.
\begin{equation}
\begin{aligned}
    \frac{\partial f^{(1)}(x;t)}{\partial t} 
    &= -  \eta \ f(x;t) + \frac{\eta}{\Vert \Omega \Vert}  \int\limits_{\Omega \setminus \{x\}} d\bar{x} \ f(\bar{x};t) \\
    &= - \eta\ f(x;t) + \frac{\eta}{\Vert \Omega \Vert} \int\limits_{\Omega} d\bar{x}\ f(\bar{x};t)  \\
    &= \eta \left( \frac{1}{\Vert \Omega \Vert} - f(x;t) \right),
    %&= \eta \left( \frac{1}{2\pi} - f(x;t) \right),
    \label{eq:continuous_pde_state_firsttype}
\end{aligned}
\end{equation}
where we used $\bar{x}$ as integration variable which is integrated over the complete set of states, except for state $x$. The first term corresponds to nodes leaving state $x$ for another uniformly chosen state and the second term to nodes changing spontaneously to state $x$. Note, since the density function integral in a point equals zero the integral over $\Omega \setminus \{x\}$ equals the integral over the complete set $\Omega$. The expression should be divided by the total size of the set $\Vert \Omega \Vert$, as the node should change to the specific state $x$. In the last step, we made use of the fact that the integral of a state density function over the complete set of states equals 1. 

It can easily be observed that the solution to \cref{eq:continuous_pde_state_firsttype} converges to the uniform distribution density function, that is $ \lim\limits_{t \to \infty} f(x;t) = \frac{1}{\Vert \Omega \Vert}$. In particular, if $f(x;t) > \frac{1}{\Vert \Omega \Vert}$ we have $\frac{\partial f(x;t)}{\partial t} < 0$ and vice versa. 

The change in link density $l_2(x,y;t)$ due to interactions of type 1 is described by the following PDE. From now on we will write $\int\limits_{\Omega \setminus \{x\}}$ directly as $\int\limits_{ \Omega}$. We have
\begin{equation}
\begin{aligned}
    \frac{\partial l_2^{(1)}(x,y;t)}{\partial t} 
    &= - 2\eta\ l_2(x,y;t) + \frac{\eta}{\Vert \Omega \Vert} \int\limits_{\Omega}dz\ l_2(z,y;t)   + \frac{\eta}{\Vert \Omega \Vert} \int\limits_{\Omega}dz\ l_2(x,z;t)    \\
 %   &=  - 2\eta\ l_2(x,y;t) + \frac{\eta}{2\pi} \int\limits_{\Omega}dz\ l_2(z,y;t)   + \frac{\eta}{2\pi} \int\limits_{\Omega}dz\ l_2(x,z;t)\\ 
    &= - 2\eta \ l_2(x,y;t) + \frac{\eta}{\Vert \Omega \Vert} \int\limits_{\Omega}dz\ \left( l_2(z,y;t)   +  l_2(x,z;t) \right)  
    ,
    \label{eq:continuous_pde_link_firsttype}
\end{aligned}
\end{equation}
in which the second and third term correspond to the creation of $x-y$-links. That is $z-x$- and $z-y$-links with an arbitrary $z \in \Omega$ changing to $y$ or $x$ respectively. The first term describes the destruction of $x-y$-links due to one of the nodes changing to another state, hence the factor~2. The last simplification step is justified by the linearity property of the Riemann integral.

Equations (\ref{eq:continuous_pde_state_firsttype}) and (\ref{eq:continuous_pde_link_firsttype}) describe a continuous system which obeys interactions of the first type only. The next step will be to obtain similar equations for the second type of dynamics. For the state density $f(x;t)$, these interactions can be described as follows,
\begin{equation}
\begin{aligned}
\frac{\partial f^{(2)}(x;t)}{\partial t} 
&= \sigma_c \int d \xi  \int\limits_{\Omega}dz\ l_3(x-\xi ,z,x+\xi ;t)   - \sigma_c \int d\xi \ \int\limits_{\Omega} dz\ l_3(z-\xi,x,z+\xi;t) \\
&= \sigma_c \int\limits_{\Omega}dz\ \int d \xi\ \left( l_3(x-\xi ,z,x+\xi ;t)   - l_3(z-\xi,x,z+\xi;t) \right)
.
\label{eq:continuous_pde_state_secondtype}
\end{aligned}
\end{equation}
The first term represents nodes in an arbitrary state $z$, in between two nodes in states $x-\xi $ and $x+\xi $ such that their average state is $x$. This way an extra $x$ node is created due to three-body interactions at rate $\sigma_c$. The second term corresponds to removal of $x$ nodes due to these interactions. We take an $x$ node in between two arbitrary nodes in states $z+\xi$ and $z-\xi$. Their average state is $z$, which is in general not equal to $x$, which justifies the integration over the complete state set $\Omega$. The simplification step is again justified by linearity of the integral and moreover by Fubini's double integral theorem \cite{Fubini1907}.
However, we should be careful, since the integral boundaries may demand function evaluations outside of the domain on which the density functions are defined. In that case, there are two different ways to compute the integrals, depending on the application. The state set can be a normal interval $\Omega =[a,b]$. Then we simply define all density distributions to be zero outside the state set $\Omega$, such that the integrals can be evaluated. Another possibility is that the state set is periodic, for example $\Omega =[-\pi,\pi)$, with $x+2\pi=x$ for all $x\in\Omega$. We want the integrals to take this periodicity into account. Hence, we would make even extensions of the density distribution functions at the boundaries of the state set. In the next part of this thesis, we will elaborate on making the system specific for application on such a periodic state set.

The next partial differential equation captures the change in the link density function $l_2(x,y;t)$ due to the three-body interactions 
\begin{equation}
\begin{aligned}
\frac{\partial l^{(3)}_2(x,y;t)}{\partial t} 
=& \phantom{+..}
   \sigma_c \int\limits_{\Omega} dz \int d\xi\ l_4(^y z^{x-\xi }_{x+\xi };t) \\
&+ \sigma_c \int\limits_{\Omega} dz \int d\xi\ l_4(^x z^{y-\xi }_{y+\xi };t)  \\
&- \sigma_c \int\limits_{\Omega} dz \int d\xi\ l_4(^y x^{z-\xi}_{z+\xi};t) \\
&- \sigma_c \int\limits_{\Omega} dz \int d\xi\ l_4(^x y^{z-\xi}_{z+\xi};t) \\
&+ \sigma_c \int\limits_{\Omega} dz\ l_3(y,z,-y+2x;t) \\
&+ \sigma_c \int\limits_{\Omega} dz\ l_3(x,z,-x+2y;t) \\
&- \sigma_c \int\limits_{\Omega} dz\ l_3(y,x,z;t) \\ 
&- \sigma_c \int\limits_{\Omega} dz\ l_3(x,y,z;t) 
.
\label{eq:continuous_pde_link_secondtype}
\end{aligned}
\end{equation}
The first term represents four-body subgraphs of configuration $^y z^{x-\xi }_{x+\xi }$ in which the middle node $z$ changes to the average state of $x-\xi $ and $x+\xi $, which is $x$. Here, one $x-y$-link is created with the neighbouring $y$ node. We integrate over all $z \in\Omega$ to include all nodes which might change to state $x$. Again, we integrate over all $\xi $ for which our averaging operation is defined properly. The second term is quite similar. Here, the positions of nodes in state $x$ and $y$ are inverted. 
The third term corresponds to four-body subgraphs of configuration $^y x^{z-\xi }_{z+\xi }$. Now, the middle node takes with rate $\sigma_c$ the average of the two neighbouring nodes $z-\xi$ and $z+\xi$, which is $z$ and does not equal $x$ in general. This causes the $y-x$-link to be changed in a $y-z$-link, explaining the minus sign. The fourth term is again similar to the third one.
Three-body subgraphs with the potential to form an $x-y$-link are taken into account in the fifth and sixth term. In triplets with configuration $y-z-(-y+2x)$ and $x-z-(-x+2y)$ the middle node takes the average value of the outer two, which is $x$ or $y$ respectively, with rate $\sigma_c$. With this, an $x-y$-link is created with the already existing $y$ or $x$ node.
Three-body subgraphs $x-y-z$ and $y-x-z$ are taken into account in the last two terms. The middle node takes the average state of the outer two, which is again in general not $x$ or $y$. We integrate over all $z\in\Omega$ to include all subgraphs of this form. Note, $l_3(x,y,z;t)=l_3(z,y,x;t) \ne l_3(y,x,z;t)$. Note too that this equation is still an approximation. There could always be more terms added. However, these terms would correspond to subgraphs with a very specific configuration. We assume that the densities of these subgraphs are small compared to the other terms, such that they can be omitted.

The next step will be adding the link dynamics. Equations (\ref{eq:continuous_pde_state_thirdtype}) and (\ref{eq:continuous_pde_link_thirdtype}) describe the link creation interactions. Since the node states do not change when adding links, we have
\begin{equation}
\begin{aligned}
    \frac{\partial f^{(3)}(x;t)}{\partial t} 
    &= 0
    .
    \label{eq:continuous_pde_state_thirdtype}
\end{aligned}
\end{equation}
Moreover, the fact that links are created between two arbitrary nodes with rate $\alpha$ is described as 
\begin{equation}
\begin{aligned}
    \frac{\partial l^{(3)}_2(x,y;t)}{\partial t} 
    &= \alpha \ f(x;t)\ f(y;t)
    .
    \label{eq:continuous_pde_link_thirdtype}
\end{aligned}
\end{equation}

The deletion of links is described by similar PDEs. Again, the node states stay unchanged, yielding
\begin{equation}
\begin{aligned}
    \frac{\partial f^{(4)}(x;t)}{\partial t} 
    &= 0,
    \label{eq:continuous_pde_state_fourthtype}
\end{aligned}
\end{equation}
whilst the removal of arbitrary links with rate $\beta $ is captured by 
\begin{equation}
\begin{aligned}
    \frac{\partial l^{(4)}_2(x,y;t)}{\partial t} 
    &= -\beta\ l_2(x,y;t)
    .
    \label{eq:continuous_pde_link_fourthtype}
\end{aligned}
\end{equation}



%%%%%%%%%%%%%%%%%%%%%%%%%%%%%%%%%%%%%%%%%%%%%%%%%%%%%%%%%%%%%%%%%%%%%%%%%%%%%%%%%%%%%%%%%




\section{Completing the continuous model}

Equations (\ref{eq:continuous_pde_state_firsttype}) - (\ref{eq:continuous_pde_link_fourthtype}) describe the influence of the four types of interactions on the time derivatives of the state and link density functions. To obtain the complete model, these PDEs can simply be added. This yields the following equation for the state density function
\begin{equation}
\begin{aligned}
	\frac{\partial f(x;t)}{\partial t}  
	=& \frac{\partial f^{(1)}(x;t)}{\partial t}  + \frac{\partial f^{(2)}(x;t)}{\partial t}  + \frac{\partial f^{(3)}(x;t)}{\partial t}  + \frac{\partial f^{(4)}(x;t)}{\partial t} \\
	=&\ \eta \left( \frac{1}{\Vert \Omega \Vert} - f(x;t) \right) \\
	&+  \sigma_c \int\limits_{\Omega}dz\  \int d \xi \ \left( l_3(x-\xi ,z,x+\xi ;t)  - l_3(z-\xi,x,z+\xi;t) \right)
	.
	\label{eq:continuous_pde_state_complete_w_ho_terms}
\end{aligned}
\end{equation}

\raggedbottom

The PDE describing the link density function can be obtained in a similar fashion
\begin{equation}
\begin{aligned}
	\frac{\partial l_2(x,y;t)}{\partial t} 
	=& \frac{\partial l^{(1)}_2(x,y;t)}{\partial t} + \frac{\partial l^{(2)}_2(x,y;t)}{\partial t} + \frac{\partial l^{(3)}_2(x,y;t)}{\partial t} + \frac{\partial l^{(4)}_2(x,y;t)}{\partial t}\\
	=&\
	\frac{\eta}{\Vert \Omega \Vert} \int\limits_{\Omega}dz\ \left( l_2(z,y;t)   +  l_2(x,z;t) \right) \\
	&+ \sigma_c \int\limits_{\Omega} dz \int d \xi \ \left( l_4(^y z^{x-\xi}_{x+\xi};t) +  l_4(^x z^{y-\xi}_{y+\xi};t) \right)  \\
	&- \sigma_c \int\limits_{\Omega} dz \int d \xi \ \left( l_4(^y x^{z-\xi}_{z+\xi};t) +  l_4(^x y^{z-\xi}_{z+\xi};t) \right) \\
	&+ \sigma_c \int\limits_{\Omega} dz\ \left( l_3(y,z,-y+2x;t) + l_3(x,z,-x+2y;t)  \right) \\
	&- \sigma_c \int\limits_{\Omega} dz\ \left( l_3(y,x,z;t)     + l_3(x,y,z;t)      \right) \\
	&+ \alpha \ f(x;t)\ f(y;t) \\
	&- (2\eta + \beta) \ l_2(x,y;t)
	,
	\label{eq:continuous_pde_link_complete_w_ho_terms}
\end{aligned}
\end{equation}
where again the linearity of the Riemann integral was used to simplify the expression. Equations (\ref{eq:continuous_pde_state_complete_w_ho_terms}) and (\ref{eq:continuous_pde_link_complete_w_ho_terms}) together describe the adaptive network model with a continuous state set in the most general form. There are two main differences with the discrete adaptive networks. Firstly, the types of dynamics had to be slightly altered. Secondly, the network evolution is described by partial (integro-)differential equations, instead of ordinary differential equations. Although these are in general harder to solve, there are only two coupled equations. For the discrete state set we needed $M(2+\frac{1}{2}(M-1))$ coupled equations for $M$ states.

Note however, that not all terms are known. Just as in the discrete adaptive networks, moment equations of a given order are dependent on higher order terms. One might think this means we need extra PDEs for these higher order moments, but then we would go on until we get equations for subgraphs of size in the order of the network size, resulting in a system of a large amount connected PDEs which would be very hard to solve. Moreover, numerical estimates would not allow for developing an analytical solution. This can be prevented in multiple ways. In the next sections, two possibilities are described: the mean field approximation and the moment closure approximation, both were also applied for the discrete state adaptive networks in the first part of this work.  

%%%%%%%%%%%%%%%%%%%%%%%%%%%%%%%%%%%%%%%%%%%%%%%%%%%%%%%%%%%%%%%%%%%%%%%%%%%%%%%%%%%%%



\section{Mean Field model}
A first approach into solving a continuous state adaptive network model would be considering a mean field approximation. This is equivalent to neglecting the link dynamics and assuming that the density of links connecting nodes in a given state is proportional to the product of the densities of these states, similar what was done to the discrete state adaptive networks. This simplifies \cref{eq:continuous_pde_state_complete_w_ho_terms} to the following mean field approximation 


\begin{equation}
\begin{aligned}
\frac{\partial}{\partial t}\ f(x;t) &= 
\eta \left( \frac{1}{\Vert \Omega \Vert} - f(x;t) \right) \\ 
&+ \sigma_c\langle k \rangle^2 \int\limits_{\Omega} dz \int\limits d\xi\
f(x-\xi;t)\ f(z,t)\ f(x+\xi;t) \\
&- \sigma_c \langle k \rangle^2 \int\limits_{\Omega} dz  \int\limits_{\Omega} dw\ f(w;t)\ f(x;t)\ f(z;t),
\end{aligned}
\end{equation}
where $\bar{x}$ is used as an integration variable and in which $\langle k \rangle$ is the aforementioned proportionality constant, representing the mean network degree. Since we are dealing with a density function, $\int\limits_{\Omega} dx\ f(x;t) =1$, such that we can simplify to
\begin{equation}
\begin{aligned}
\frac{\partial}{\partial t}\ f(x;t) &= 
\eta \left( \frac{1}{\Vert \Omega \Vert} - f(x;t) \right) \\
&	+ \sigma_c\langle k \rangle^2  \int\limits d\xi\
f(x-\xi;t)\ f(x+\xi;t) \\
&	- \sigma_c \langle k \rangle^2\ f(x;t) \int\limits_{\Omega} dz \int\limits d\xi\ f(z-\xi;t)\ f(z+\xi;t).
\end{aligned}
\label{eq:cont_state_mean_field}
\end{equation}
Since link dynamics are neglected in the mean field approximation there is no need to approximate \cref{eq:continuous_pde_link_complete_w_ho_terms}. Therefore, the mean field system is described with only one partial integro-differential equation, which has the advantage that it allows for easier analysis compared to a system of PDEs. In \cref{chapter:swarming_systems} we apply the mean field model to two-dimensional swarming motion of self-propelled particles.

\section{Moment closure approximation}
In order to obtain the system of two coupled, closed PDEs we seek expressions for the unknown terms $l_3(x,y,z;t)$ and $l_4(^x y^z_w;t)$ in terms of the known lower order moments $l_2(x,y;t)$ and $f(x;t)$. Analogous to the discrete state adaptive network models, the closure used will be the pair level closure, 
\begin{align}
l_3(x,y,z;t) & = \frac{l_2(x,y;t)\ l_2(y,z;t)}{f(y;t)},\\
l_4(^x y^z_w;t) &= \frac{l_2(x,y;t)\ l_2(y,z;t)\ l_2(y,w;t)}{f(y;t)^2},	
\end{align}
which is elaborated on in the first part of this thesis. Additional derivation and explanation can be found in \cite{Chen2016, Demirel2014, Newman2003}. We will assume that this closure relation is still valid in the continuous state case.
Applying the approximation would result in a closed system of two coupled PDEs, one for the state density function $f(x;t)$ and one for the first order link density function $l_2(x,y;t)$. Hence, it will be convenient to write  $l_2(x,y;t) =  l(x,y;t)$ from now on. With this approximation, the adaptive network models with a continuous state set are captured by the following two coupled non-linear partial integro-differential equations

\begin{subequations}\label{eq:cont_complete}
\small
\begin{adjustwidth}{-.5in}{-.5in} 
	\begin{alignat}{2}
	\frac{\partial f(x;t)}{\partial t} = &
	\ \eta \left( \frac{1}{\Vert \Omega \Vert} - f(x;t) \right) \\
	& + \sigma_c\int\limits_{\Omega} dz \int\limits  d \xi  \
	\Bigg[ 
	\frac{l(x-\xi ,z;t)\ l(z,x+\xi ;t)}{f(z;t)} - \frac{l(z-\xi ,x;t)\ l(x,z+\xi ;t)}{f(x;t)} 
	\Bigg]  \nonumber \\[15pt] 
	\frac{\partial l(x,y;t)}{\partial t} = &
	\int\limits_{\Omega} dz \
	\Bigg{\{ }
	\frac{\eta}{\Vert \Omega \Vert} \left( l(x,z;t) + l(z,y;t) \right)   \\
	& \phantom{\int\limits_{\Omega} dz\ \sum}
	+\sigma_c \frac{l(y,z;t)\ l(z,-y+2x;t)}{f(z;t)}
	+\sigma_c \frac{l(x,z;t)\ l(z,-x+2y;t)}{f(z;t)} \nonumber \\
	& \phantom{\int\limits_{\Omega} dz\ \sum}
	-\sigma_c \frac{l(y,x;t)\ l(x,z;t)}{f(x;t)} 	 
	-\sigma_c \frac{l(x,y;t)\ l(y,z;t)}{f(y;t)} \nonumber \\
	& \phantom{\int\limits_{\Omega} dz\ \sum}
	+ \sigma_c\int\limits d \xi \    
	\Bigg[  
	\frac{l(y,z;t)\ l(z, x-\xi;t)\ l(z, x+\xi;t)}{f(z;t)^2}  + 
	\frac{l(x,z;t)\ l(z, y-\xi;t)\ l(z,y+\xi ;t)}{f(z;t)^2} \nonumber  \\ 
	& \phantom{\int\limits_{\Omega} dz\ \sum \sigma_c\int\limits \ d \xi \sum }
	- \frac{l(y,x;t)\ l(x, z-\xi;t)\ l(x,z+\xi ;t)}{f(x;t)^2} - 
	\frac{l(x,y;t)\ l(y, z-\xi ;t)\ l(y, z+\xi ;t)}{f(y;t)^2} 
	\Bigg]	 \
	\Bigg{ \} } \nonumber \\
	&+ \alpha\ f(x;t)\ f(y;t) - 
	(2\eta+\beta)\ l(x,y;t).\nonumber
	\end{alignat}
\end{adjustwidth}
\end{subequations}
\normalsize

