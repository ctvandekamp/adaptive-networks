\chapter{Abstract}
We consider adaptive network models with discrete and continuous state sets obeying dynamical rules that enable application to swarming systems. The 2-state adaptive network contains a supercritical pitchfork bifurcation in the transition between ordered and disordered stationary solutions. We derive an adaptive network model that works on a continuous state set and apply it to swarming motion in both a mean field and a moment closure approximation. In numerical solutions of the mean field approximation the relation between the variance of the ordered stationary distributions and the system parameters is given by a square root function. Cauchy distributions form a good fit to these steady state distributions, although they are not the analytic stationary solutions. We show that in numerical solutions of the moment closure approximation a bistable region is formed, in which the initial condition determines if the system ends up in an ordered or a disordered state. Further research could focus on finding the exact details of the corresponding subcritical pitchfork and saddle-node bifurcations and comparing the derived models to real-life swarming systems.