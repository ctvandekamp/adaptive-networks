%%%% PREAMBLE

% TEXT

\usepackage{libertine}									% Font
\usepackage{libertinust1math}							% Math font
\usepackage[scaled=0.8]{beramono}						% Monospace font

\usepackage[T1]{fontenc}                                % Makes it possible to use glyphs like "o etc
\usepackage[UKenglish]{babel}							% Hyphenation according to UK rules


%%%%%%%%%%%%%%%%%%%%%%%%%%%%%%%%%%%%%%%%%%%%%%%%%%%%%%%%%%%%%%%%%%%%%%%%%%%%%%%%%%%%%%
% MATH
 
\usepackage{amsmath, amsfonts, amsthm, bm}
\usepackage{mathabx}									% for \bigtimes for cartesian products
\DeclareMathOperator\erf{erf}  							% error function as \erf

%%%%%%%%%%%%%%%%%%%%%%%%%%%%%%%%%%%%%%%%%%%%%%%%%%%%%%%%%%%%%%%%%%%%%%%%%%%%%%%%%%%%%%
% LINKS, REFERENCES, CITING ETC

\usepackage[colorlinks=false,hidelinks,pdfpagelabels]{hyperref}
\hypersetup{bookmarks=true}
\usepackage[noabbrev]{cleveref}		% Makes clever referencing using \cref possible.

%%%%%%%%%%%%%%%%%%%%%%%%%%%%%%%%%%%%%%%%%%%%%%%%%%%%%%%%%%%%%%%%%%%%%%%%%%%%%%%%%%%%%%
% STYLE

\renewcommand{\cftchapterfont}{\normalfont}				% changes the font of sections in the TOC
\renewcommand{\cftchapterpagefont}{\normalfont}			% changes the font of sections in the TOC
\renewcommand*{\cftchapterleader}{\normalfont\cftdotfill{\cftsectiondotsep}}		% add dotted line for chapters to TOC

\DisemulatePackage{setspace}							% makes setspace compatible with memoir
\usepackage[labelfont={bf},textfont=it]{caption}		% makes captions bold + italic
\usepackage{setspace}									% increase line spacing
\captionsetup{font={stretch=1.25}}      				% change 1.25 for figures and tables
\renewcommand{\baselinestretch}{1.25}					% increase line spacing such that inline formulae fit better


\makechapterstyle{carsten}{%
	\setlength\beforechapskip{0pt}
	\setlength\midchapskip{0pt}
	\setlength\afterchapskip{40pt}
	\renewcommand\chapnamefont{%
		\normalfont\centering\large\scshape\MakeLowercase}
	\renewcommand\chapnumfont{%
		\normalfont\centering\fontsize{60pt}{0pt}\selectfont}
	\renewcommand\chaptitlefont{%
		\normalfont\HUGE\bfseries\centering}
	\renewcommand\printchaptername{%
		\marginpar{\chapnamefont{\@chapapp}}}
	\renewcommand\chapternamenum{}
	\renewcommand\printchapternum{%
		\marginpar{\chapnumfont\thechapter}}
}

% PAGE MARGINS
\usepackage{geometry} 									% for setting other page margins in title page
\setulmarginsandblock{3cm}{3cm}{*}		
\setlrmarginsandblock{2cm}{4cm}{*}
\checkandfixthelayout


\usepackage{titlesec}				
\titlespacing*{\section}{0pt}{1cm}{.5cm}  				% specificeren ruimte rondom sections, {left}{above}{under}


\usepackage[bottom]{footmisc} 							% sets footnotes to bottom of page



%% Floats & captions

\usepackage{subcaption} 								% makes subtables possible



\theoremstyle{plain}
\newtheorem{theorem}{Theorem}[chapter] 					% reset theorem numbering for each chapter
\newtheorem{definition}[theorem]{Definition} 			% definition numbers are dependent on theorem numbers
\newtheorem{example}[theorem]{Example} 					% same for example numbers

\def\code#1{\texttt{#1}}                                % Redefine \code command for mono-spaced text


%%%%%%%%%%%%%%%%%%%%%%%%%%%%%%%%%%%%%%%%%%%%%%%%%%%%%%%%%%%%%%%%%%%%%%%%%%%%%%%%%%%%%%
% FIGURES

\usepackage{graphicx}


%%%%%%%%%%%%%%%%%%%%%%%%%%%%%%%%%%%%%%%%%%%%%%%%%%%%%%%%%%%%%%%%%%%%%%%%%%%%%%%%%%%%%%
% TIKZ

\usepackage{tikz}
\usetikzlibrary{quotes,angles}
\usepackage{pgfplots}

% Define dotted pattern to fill nodes
\usetikzlibrary{arrows,patterns}
\pgfdeclarepatternformonly{soft crosshatch}{\pgfqpoint{-1pt}{-1pt}}{\pgfqpoint{4pt}{4pt}}{\pgfqpoint{3pt}{3pt}}%
{
	\pgfsetstrokeopacity{0.3}
	\pgfsetlinewidth{0.4pt}
	\pgfpathmoveto{\pgfqpoint{3.1pt}{0pt}}
	\pgfpathlineto{\pgfqpoint{0pt}{3.1pt}}
	\pgfpathmoveto{\pgfqpoint{0pt}{0pt}}
	\pgfpathlineto{\pgfqpoint{3.1pt}{3.1pt}}
	\pgfusepath{stroke}
}


%%%%%%%%%%%%%%%%%%%%%%%%%%%%%%%%%%%%%%%%%%%%%%%%%%%%%%%%%%%%%%%%%%%%%%%%%%%%%%%%%%%%%%
% OTHER
\newcommand\todo[1]{\textcolor{red}{#1}}



\newcommand\X{[X]}
\newcommand\Y{[Y]}
\newcommand\Yi{[Y_i]}
\newcommand\XX{\mathit{[XX]}}
\newcommand\YY{\mathit{[YY]}}
\newcommand\XY{\mathit{[XY]}}
\newcommand\YZ{\mathit{[YZ]}}
\newcommand\ZY{\mathit{[ZY]}}
\newcommand\XZ{\mathit{[XZ]}}
\newcommand\WX{\mathit{[WX]}}
\newcommand\YW{\mathit{[YW]}}
\newcommand\XYZ{\mathit{[XYZ]}}
\newcommand\XYX{\mathit{[XYX]}}
\newcommand\XYY{\mathit{[XYY]}}
\newcommand\XXX{\mathit{[XXX]}}
\newcommand\XYZW{\mathit{[^XY^Z_W]}}

\renewcommand{\Re}{\text{}\operatorname{Re}}
\renewcommand{\Im}{\text{}\operatorname{Im}}